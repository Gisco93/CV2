%=========================================================

% Here you can choose to compile with or without solutions.
% However, this definition is ignored if you use any
% command from the `Makefile`.
\providecommand{\withSol}{\iftrue}

%=========================================================

\documentclass
[twoside,german,colorbacktitle,accentcolor=tud9c]
{tudexercise}

\usepackage[T1]{fontenc}
\usepackage[utf8]{inputenc}
\usepackage[ngerman]{babel}
\usepackage{amstext}
\usepackage{amsmath}
\usepackage{graphicx}
%\usepackage{setspace}
\usepackage{multicol}
\usepackage{mathtools}
\usepackage{dsfont}
\usepackage{units}
%\usepackage{subfigure}
\usepackage{color}
\usepackage{booktabs}
\usepackage{fancyref}
\usepackage{gensymb}
\usepackage{tikz}
\usetikzlibrary{shapes.misc} 	
\usepackage[verbose]{placeins}%Floatbarrier
\usepackage{tikz}
\usetikzlibrary{bayesnet}


%=========================================================


\setcounter{section}{2}
%=========================================================

\newcommand{\grp}{8}

%=========================================================


\begin{document}

\title{Computer Vision 2\\ Assignment  \arabic{section}}
\subtitle{Summer Semester 2019}
\subsubtitle{Group \grp{} \\ Moritz Fuchs $\&$ Diedon Xhiha}

\maketitle

%=========================================================



\begin{examheader}
	\textmb{Computer Vision 2 \\  Assignment  \arabic{section} | Group \grp{}}\\
	Moritz Fuchs	$\&$ Diedon Xhiha	\\ 
\end{examheader} 


%=========================================================
% Anpassung an Aufgabenstellung
\renewcommand\thesubsection{Problem \arabic{subsection}}
\renewcommand\thesubsubsection{\arabic{subsubsection}.}

%=========================================================
\FloatBarrier
\newif\ifvimbug
\vimbugfalse

\ifvimbug
\begin{document}
\fi


\subsection{Graphical models (20 Points)}
\subsubsection{1 Point}
Two main ingredients the set of random variables describing the entities involved the problem, and the second ingredient is the set of conditional probabilities, that tell us the relation between a certain variable and another variable or variables from the set.
\subsubsection{1 Point}
As we know that directed graphs are versatile, but not always appropriate. So they are not always be convenient  to provide conditional distributions, and some of certain conditional independence structures that a directed graph can not represent. So for example \textit{Loopy graph}, it is not possible to express the same conditional independence statements using the a directed graph model.
\subsubsection{2 Points}
	
\begin{tikzpicture}
%  \tikz{
% nodes
     \node[latent] (x_6) {$x_6$} ; %
     \node[obs,above=of x_6,xshift=-1cm,path picture={\ (path picture bounding box.south) rectangle (path picture bounding box.north west);}] (x_4) {$x_4$}; %
     \node[obs,above=of x_6,xshift=1cm,path picture={\ (path picture bounding box.south) rectangle (path picture bounding box.north west);}] (x_5) {$x_5$}; %
     \node[obs,above=of x_4,path picture={\ (path picture bounding box.south) rectangle (path picture bounding box.north west);}] (x_2) {$x_2$}; %
     \node[obs,above=of x_2,xshift=-1cm,path picture={\ (path picture bounding box.south) rectangle (path picture bounding box.north west);}] (x_1) {$x_1$}; %
     \node[obs,above=of x_2,xshift=1cm,path picture={\ (path picture bounding box.south) rectangle (path picture bounding box.north west);}] (x_3) {$x_3$}; %
     \node[obs,above=of x_2,xshift=2cm,path picture={\ (path picture bounding box.south) rectangle (path picture bounding box.north west);}] (x_7) {$x_7$}; %
% plate
    \plate [inner sep=.25cm,yshift=.2cm] {plate1} {(x_1)(x_2)(x_3)(x_4)(x_5)(x_6)(x_7)} {$Figure 1$}; 
% edges
     \edge {x_4,x_5} {x_6} 
     \edge {x_2} {x_4}  
     \edge {x_2} {x_5}  
     \edge {x_1,x_3} {x_2}
     \edge {x_7} {x_5}
%}
\end{tikzpicture}

The Markov Blanket of variable $x_2$ is formed from the variables that are filled with gray color.

\subsubsection{2 Points}
The factorization of the directed graphical model is as follows\\
$p(x_1,x_2,x_4,x_5,x_6,x_7,x_8,x_9,x_{10},x_{11},x_{12},x_{13},x_{14},x_{15})=p(x_1)p(x_2)p(x_3)p(x_6)p(x_7)p(x_4|x_1,x_2)p(x_5|x_2,x_3)p(x_{10}|x_7)\\p(x_{11}|x_{10})p(x_8|x_4,x_5)p(x_9|x_3,x_5,x_6)p(x_14|x_{11})p({x_12}|x_8,x_9)p(x_{13}|x_9)p(x_{15}|x_{12})$\\ \\
The factorization of the undirected graphical model is as follows\\
$p(x_1,x_2,x_4,x_5,x_6,x_7,x_8,x_9,x_{10},x_{11},x_{12},x_{13},x_{14},x_{15})=\frac{1}{Z}\phi_0(x_3) \phi_1(x_1,x_4) \phi_2(x_2,x_4,x_5,x_8) \phi_3(x_7,x_{10}) \phi_4(x_7,x_{11})\\ \phi_5(x_{11},x_{14}) \phi_6(x_5,x_9) \phi_7(x_8,x_{12}) \phi_8(x_9,x_{12},x_{13},x_{15}) \phi_9(x_6,x_9)$
\subsubsection{1 Point}
---- excluded----
\begin{tikzpicture}
%  \tikz{
% nodes
     \node[latent] (x_6) {$x_6$} ; %
     \node[latent,above=of x_6,path picture={\ (path picture bounding box.south) rectangle (path picture bounding box.north west);}] (x_5) {$x_5$}; %
     \node[latent,above=of x_6,xshift=4cm,path picture={\ (path picture bounding box.south) rectangle (path picture bounding box.north west);}] (x_3) {$x_3$}; %
     \node[latent,above=of x_5,xshift=-1cm,path picture={\ (path picture bounding box.south) rectangle (path picture bounding box.north west);}] (x_1) {$x_1$}; %
     \node[latent,above=of x_5,xshift=1cm,path picture={\ (path picture bounding box.south) rectangle (path picture bounding box.north west);}] (x_2) {$x_2$}; %
     \node[latent,above=of x_5,xshift=4cm,path picture={\ (path picture bounding box.south) rectangle (path picture bounding box.north west);}] (x_4) {$x_4$}; %

% plate
    \plate [inner sep=.25cm,yshift=.2cm] {plate1} {(x_1)(x_2)(x_3)(x_4)(x_5)(x_6)} {$Figure 2$}; 
% edges
     \draw 
     (x_1) -- (x_2)
     (x_1) -- (x_5)
     (x_5) -- (x_6) 
     (x_2) -- (x_4)
     (x_2) -- (x_3) 
     (x_4) -- (x_3) 
     (x_2) -- (x_5);
   \draw[double](x_5) -- (x_1);
%}
\end{tikzpicture}

\subsubsection{2 Points}
  \begin{tabular}{ | l | l | l | l | l | l | }
	    \hline
$x_1$ &  $ x_2$  &  $x_3$  &  $x_4$  & $p(x_1,x_2,x_3,x_4)$  & 	$\frac{1 }{Z}p(x_1,x_2,x_3,x_4)$			\\ \hline
0	&   0	 &	0  &  	0   &    1*1*1*1=1              & 	0.1077			\\ \hline
0	&   0	 &	0  & 	1   &    1*1*0.1*0.1=0.01	    & 	0.001077			\\ \hline
0	&   0	 &	1  &	0   &    1*0.1*0.1*1=0.01		& 	0.001077			\\ \hline
0	&   0	 &	1  &	1   &    1*0.1*2*0.1=0.02       & 	0.002155			\\ \hline
0	&   1	 &	0  &	0   &    0.1*0.1*1*1=0.01       & 	0.001077			\\ \hline
0	&   1	 &	0  &	1   &    0.1*0.1*0.1*0.1=0.0001 & 	0.00001077			\\ \hline
0	&   1	 &	1  &	0   &    0.1*2*0.1*1=0.02       & 	0.002155			\\ \hline
0	&   1	 &	1  &	1   &    0.1*2*2*0.1=0.04       & 	0.00431			\\ \hline
1	&   0	 &	0  &	0   &    0.1*1*1*0.1=0.01       & 	0.001077			\\ \hline
1	&   0	 &	0  &	1   &    0.1*1*0.1*2=0.02       & 	0.002155			\\ \hline
1	&   0	 &	1  &	0   &    0.1*0.1*0.1*0.1=0.0001 & 	0.00001077			\\ \hline
1	&   0	 &	1  &	1   &    0.1*0.1*2*2=0.04       & 	0.00431			\\ \hline
1	&   1	 &	0  &	0   &    2*0.1*1+0.1=0.02       & 	0.002155			\\ \hline
1	&   1	 &	0  &	1   &    2*0.1*0.1*2=0.04       & 	0.00431			\\ \hline
1	&   1	 &	1  &	0   &    2*2*0.1*0.1=0.04       & 	0.00431			\\ \hline
1	&   1	 &	1  &	1   &    2*2*2*2=8              & 	0.86206			\\ \hline

  \end{tabular}
\\\\
We have computed Z as the sum of all the probabilities above, and after the calculation we found Z=9.28.
\subsubsection{2 Points}
Based on those four records \\ \\
  \begin{tabular}{ | l | l | l | l | l |}
	    \hline
a &  b  &  c &   p(a,b,c)  \\ \hline
0	&   1	 &	0  & 0.048              \\ \hline
1	&   0	 &	0  &  0.192	          \\ \hline
0	&   1	 &	1  &  0.216	          \\ \hline
1	&   0	 &	1  &  0.064              \\ \hline

  \end{tabular}
\\ \\
The first and second row c remain the same but a and b change and with them the distribution too, the same happens in the third and fourth row. So we can conclude that those variable a and b are marginally dependent, when b=1 and a=0 then the distribution ia 0.048 but in the other hand, when b=0 and a=1 then the distribution is 0.192, so this means  knowing event b does help in value of event a and vise versa. \\ \\
\subsubsection{3 Points}

Based on the given distributions $p(a)$ $p(b|c)$ $p(c|a)$ we can draw the directed graph as follows :\\ \\
\begin{tikzpicture}
%  \tikz{
% nodes
     \node[latent] (a) {$a$} ; %
     \node[latent,xshift=5cm,path picture={\ (path picture bounding box.south) rectangle (path picture bounding box.north west);}] (b) {$b$}; %
     \node[latent,xshift=2.5cm,path picture={\ (path picture bounding box.south) rectangle (path picture bounding box.north west);}] (c) {$c$}; %

% plate
    \plate [inner sep=.25cm,yshift=.2cm] {plate1} {(a)(b)(c)} {$Figure 3$}; 
% edges
     \edge {a} {c} 
     \edge {c} {b}  
%}
\end{tikzpicture} \\
And from the figur above we can write:\\
$p(a,b,c)=p(a)p(a|c)p(b|a,c)
	    =p(a)p(a|c)p(b|c)$ \\
As we know from directed graph, because 'c' tell us about 'a', we can take away 'a' from $p(b|a,c)$, and therefore we get the desired probability.  


\subsubsection{2 Points}
Markov blanket of variables in undirected graph:\\ 
$x_1->x_2,x_3$\\
$x_2->x_1,x_4$\\
$x_3->x_1,x_4$\\
$x_4->x_2,x_3$\\ \\
Markov blanket of variables in directed graph:\\
$x_1->x_2,x_3$\\
$x_2->x_1,x_3,x_4$\\
$x_3->x_1x_2,x_4$\\
$x_4->x_2,x_3$


\subsubsection{4 Points}
Based also in the previous question about for Markov blanket for each of the variables, we can conclude that in the figure 2, undirected graph, the first one that says $x_1$ is independent of $x_4$ and that $x_2$ and $x_3$ are the Markov blanket of $x_1$, is completely true based also in the definition that, a node is conditionally independent of all other nodes given its Markov blanket.\\ 
Further mathematical proof. We have the factorization of the undirected graph: $ p(x_1,x_2,x_3,x_4) = \frac{1}{z}\phi(x_1,x_2)\phi(x_2,x_4)\phi(x_4,x_3)\phi(x_1,x_3)$.
$$ p( x_4 | x_1,x_2,x_3) = \frac{p(x_1,x_2,x_3,x_4)}{p(x_1, x_2,x_3)}$$
$$= \frac{p(x_1,x_2,x_3,x_4)}{\sum_{x_4}p(x_1, x_2,x_3,x_4)}$$
$$= \frac{ \frac{1}{z}\phi(x_1,x_2)\phi(x_2,x_4)\phi(x_4,x_3)\phi(x_1,x_3)}{\sum_{x_4} \frac{1}{z}\phi(x_1,x_2)\phi(x_2,x_4)\phi(x_4,x_3)\phi(x_1,x_3)}$$
$$= \frac{ \phi(x_1,x_2)\phi(x_2,x_4)\phi(x_4,x_3)\phi(x_1,x_3)}{\phi(x_1,x_2)\phi(x_1,x_3) \sum_{x_4} \phi(x_2,x_4)\phi(x_4,x_3)}$$
$$= \frac{ \phi(x_2,x_4)\phi(x_4,x_3)}{\sum_{x_4} \phi(x_2,x_4)\phi(x_4,x_3)}$$
$$= \frac{ \phi(x_2,x_4)\phi(x_4,x_3)\sum_{x_1} \frac{1}{z} \phi(x_1,x_3)\phi(x_1,x_2)}{\sum_{x_4} \phi(x_2,x_4)\phi(x_4,x_3)\sum_{x_1} \frac{1}{z} \phi(x_1,x_3)\phi(x_1,x_2)}$$
$$= \frac{\sum_{x_1} \frac{1}{z} \phi(x_2,x_4)\phi(x_4,x_3) \phi(x_1,x_3)\phi(x_1,x_2)}{\sum_{x_4} \sum_{x_1} \frac{1}{z}\phi(x_2,x_4)\phi(x_4,x_3) \phi(x_1,x_3)\phi(x_1,x_2)}$$
$$= \frac{\sum_{x_1} p(x_1,x_2,x_3,x_4) }{\sum_{x_4} \sum_{x_1} p(x_1,x_2,x_3,x_4) }$$
$$= \frac{p(x_2,x_3,x_4) }{p(x_2,x_3) }$$
$$= p(x_4 |x_2, x_3)$$
 The second one that says $x_2$ is independent of $x_3$, because the $x_3$ is not in the Markov blanket of $x_2$, and from definition a node is conditionally independent of all other nodes given its Markov blanket, and that $x_1$ and $x_4$ are the Markov blanket of $x_2$, because they are direct node of $x_2$.\\
Further mathematical proof (As the first one):\\
 $$ p( x_3 | x_1,x_2,x_4) = \frac{p(x_1,x_2,x_3,x_4)}{p(x_1, x_2,x_4)}$$
$$= \frac{p(x_1,x_2,x_3,x_4)}{\sum_{x_3}p(x_1, x_2,x_3,x_4)}$$
$$= \frac{ \frac{1}{z}\phi(x_1,x_2)\phi(x_2,x_4)\phi(x_4,x_3)\phi(x_1,x_3)}{\sum_{x_3} \frac{1}{z}\phi(x_1,x_2)\phi(x_2,x_4)\phi(x_4,x_3)\phi(x_1,x_3)}$$
$$= \frac{ \phi(x_1,x_2)\phi(x_2,x_4)\phi(x_4,x_3)\phi(x_1,x_3)}{\phi(x_1,x_2)\phi(x_2,x_4) \sum_{x_3} \phi(x_4,x_3)\phi(x_1,x_3)}$$
$$= \frac{\phi(x_4,x_3)\phi(x_1,x_3)}{\sum_{x_3} \phi(x_4,x_3)\phi(x_1,x_3)}$$
$$= \frac{\phi(x_4,x_3)\phi(x_1,x_3)\sum_{x_2} \frac{1}{z} \phi(x_1,x_2)\phi(x_2,x_4)}{\sum_{x_3} \phi(x_4,x_3)\phi(x_1,x_3)\sum_{x_2} \frac{1}{z} \phi(x_1,x_2)\phi(x_2,x_4)}$$
$$= \frac{\sum_{x_2} \frac{1}{z} \phi(x_4,x_3)\phi(x_1,x_3) \phi(x_1,x_2)\phi(x_2,x_4)}{\sum_{x_3} \sum_{x_2} \frac{1}{z} \phi(x_4,x_3)\phi(x_1,x_3) \phi(x_1,x_2)\phi(x_2,x_4)}$$
$$= \frac{\sum_{x_2} p(x_1,x_2,x_3,x_4) }{\sum_{x_3} \sum_{x_2} p(x_1,x_2,x_3,x_4) }$$
$$= \frac{p(x_1,x_3,x_4) }{p(x_1,x_4) }$$
$$= p(x_3 |x_1, x_4)$$
 
And for the directed Graph, first one that says $x_1$ independent of $x_4$, exactly as the definition from above, and that $x_2$ and $x_3$ are the Markov blanket of $x_1$, because $x_1$ has 2 children and no other parents and also those children have just one parent and its $x_1$. 
Further mathematical proof. We have the factorization of the directed graph: $ p(x_1,x_2,x_3,x_4) = p(x_1)p(x_2|x_1)p(x_3|x_1)p(x_4|x_3, x_2)$.
$$p(x_4|x_1,x_2,x_3) = \frac{p(x_1,x_2,x_3,x_4)}{p(x_1,x_2,x_3)}$$
$$ = \frac{p(x_1,x_2,x_3,x_4)}{\sum_{x_4} p(x_1,x_2,x_3, x_4)}$$
$$ = \frac{p(x_1)p(x_2|x_1)p(x_3|x_1)p(x_4|x_3, x_2)}{\sum_{x_4}p(x_1)p(x_2|x_1)p(x_3|x_1)p(x_4|x_3, x_2)}$$
$$ = \frac{p(x_1)p(x_2|x_1)p(x_3|x_1)p(x_4|x_3, x_2)}{p(x_1)p(x_2|x_1)p(x_3|x_1)\sum_{x_4}p(x_4|x_3, x_2)}$$
$$ = \frac{p(x_4|x_3, x_2)}{\sum_{x_4}p(x_4|x_3, x_2)}$$
$$ = \frac{p(x_4|x_3, x_2)}{1} = p(x_4|x_3, x_2)$$
Also $p(x_1|x_2,x_3,x_4) = p(x_1|x_2.x_3)$ can be proofen.\\


Last  Equation: 
$$p(x_3, x_2| x_1) = \frac{p(x_1,x_2,x_3)}{p(x_1)}$$
$$= \frac{\sum_{x_4} p(x_1,x_2,x_3,x_4)}{p(x_1)}$$
$$= \frac{\sum_{x_4} p(x_1)p(x_2|x_1)p(x_3|x_1)p(x_4|x_3, x_2)}{p(x_1)}$$
$$= \frac{p(x_1)p(x_2|x_1)p(x_3|x_1)\sum_{x_4} p(x_4|x_3, x_2)}{p(x_1)}$$
$$= p(x_2|x_1)p(x_3|x_1)\sum_{x_4} p(x_4|x_3, x_2)$$
$$= p(x_2|x_1)p(x_3|x_1) * 1$$




















\FloatBarrier
\newif\ifvimbug
\vimbugfalse

\ifvimbug
\begin{document}
\fi


\subsection{Markov random fields with Student-t potentials (10 Points)}
Tasks 1-4 within code
\setcounter{subsubsection}{1}
\subsubsection{1 Point}
As printed out:
gt log prior:                   $-9928.589720178397$\\
\subsubsection{2 Points}
As printed out:
random disparity log prior:    $ -478661.4937203116$\\
\subsubsection{2 Points}
As printed out:
constant disparity log prior:  $ 0.0$\\
\subsubsection{3 Points}
As we use MRF model, the potentialfunction expresses the compatibility between neighour pixels.\\
Therefore our random map has the lowest score as most neighbours don't fit each ohter.\\ 
The log probability of constant map ist the lowest, because neigbouring pixels are as close together as they can get and we want them to be localy smooth.\\
The ground truth lies between those log- priors as its not constant, therefore not maximally locally smooth and may have some small artifacts.\\

\FloatBarrier
\newif\ifvimbug
\vimbugfalse

\ifvimbug
\begin{document}
\fi


\subsection{Stereo with gradient-based optimization (17 Points)}
Everything else is in code.
\subsubsection{4 Points}
When initailized with a constant disparity map the struggels to find another map which does not increase the function value as the prior increases in absolut value. It therefore only detects the edges which are farest in the backgroud and the nearest on the head. The rest stays constant.\\
The random disparity does not converge to something meaningful as its to far away from such a solution, as neighbouring pixels are incompatible. It can be noticed that so of the more compatible pixel merge together but to complete this would take very long.\\
The ground truth converges to something more blurry until it gets closer to the constant map and therefore loses it usefulness. This is probably because of the brightness constancy assumption we modeled in our prior which has the highest score when its constant and therefore our gradient wants to converge to a smoother map, which very smooth and does not account for object boundaries.\\



\FloatBarrier
\newif\ifvimbug
\vimbugfalse

\ifvimbug
\begin{document}
\fi


\subsection{Stereo with a generalized robust function (10 Points)}
\setcounter{subsubsection}{3}
Everything else is in code.
\subsubsection{To obtain better results, also apply an image pyramid similar to Problem 3. (? Point/s)}
Tuned the parameters to fit the normal algorithmen and therefore doesn't show better results.\\
\subsubsection{Based on your experiments, what recommendations can you give for choosing the parameters $\alpha$ and $c$? How does each parameter influence the result? How did you go about finding good parameters (?? Point/s)}
For prior:  $\alpha = 0.5$ and $c=10.0$ or ($\alpha = 3.0$ and $c=3.0$)\\
For likelihood: $\alpha = 0.5$ and $c=10.0$ or  ($\alpha = 3.0$ and $c=4.0$)\\

$c$ does manly influences how much influence an outliner has and therefore the results is smoothed. This limits the amount of detail we can e.g. from the writing on the white board. As it is small values(e.g. 1) smooth more and bigger value (e.g. 10) for sharper edges\\
$\alpha$ efects seems quite small. Result with smaller values seem to have less depth to it, but also less grainyness.\\

We just started playing around with alpha and keep c fixed and then switch it. repeat sometimes but didnt have anything bigger strategy than our intuition.\\

\subsubsection{In general, using the generalized robust function and its gradients will not allow us to learn the shape parameters which is why we tune them manually. Why is it that gradient-based optimization using Eqs. (8) and (9) is not applicable here to obtain good shape parameters?(?? Point/s)}
The proposed loss function approximates many different loss function and interpolates inbetween them. This means we would need to fix $\alpha$ to find the perfect $c$, and in next turn fix $c$ and do the same for $\alpha$. which means we would have the perfect $c$ anymore. This result is because its a non convex function and we may only get stuck in a local optima.\\ It is better to use the different properties of the incorpated loss functions by change the $\alpha$ between iteration and do as thex describe: we can initialize $\alpha$  such that our loss is convex and then gradually reduce $\alpha$  (and therefore reduce convexity and increase robustness) during optimization, thereby enabling robust estimation that (often) avoids local minima.\\

%=========================================================

\end{document}
