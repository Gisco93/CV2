\newif\ifvimbug
\vimbugfalse

\ifvimbug
\begin{document}
\fi


\subsection{Markov random fields with Student-t potentials (10 Points)}
Tasks 1-4 within code
\setcounter{subsubsection}{1}
\subsubsection{1 Point}
As printed out:
gt log prior:                   $-9928.589720178397$\\
\subsubsection{2 Points}
As printed out:
random disparity log prior:    $ -478661.4937203116$\\
\subsubsection{2 Points}
As printed out:
constant disparity log prior:  $ 0.0$\\
\subsubsection{3 Points}
As we use MRF model, the potentialfunction expresses the compatibility between neighour pixels.\\
Therefore our random map has the lowest score as most neighbours don't fit each ohter.\\ 
The log probability of constant map ist the lowest, because neigbouring pixels are as close together as they can get and we want them to be localy smooth.\\
The ground truth lies between those log- priors as its not constant, therefore not maximally locally smooth and may have some small artifacts.\\